\documentclass[a4paper,12pt]{article}
\usepackage[margin=0.8in]{geometry}
\usepackage[slovene]{babel}
\usepackage[utf8]{inputenc}
\usepackage{lmodern}
\usepackage[T1]{fontenc}
\usepackage{eurosym}
\usepackage{graphicx}
\usepackage{amsfonts}
\usepackage{amssymb}
\usepackage{amsmath}
\usepackage{hyperref}
\usepackage{amsthm}
%\usepackage{mathpartir}
%\usepackage{stmaryrd}
%\usepackage{newtxmath}
%\usepackage{ifthen}
\usepackage{xcolor}
%\usepackage{epigraph}
%\usepackage{tikz-cd}
%\usepackage{float}
%\usepackage{framed}
%\usepackage{caption}
\usepackage{tcolorbox}
\usepackage{enumitem}


\newcommand{\todo}[1]{{\color{red}{#1}}}

\begin{document}
\title{2. DOMAČA NALOGA}
\author{Katarina Černe, 27172036}
\maketitle

\section*{1.\ naloga}

Implementacija 1.\ naloge se nahaja v datotekah \texttt{EulerImplicitna.m}, \texttt{EulerIzboljsana.m} in \texttt{Heunova.m}, rezultati pa se preverijo v datoteki \texttt{preverjanje1.m}.
Implementiramo tri različne Eulerjeve metode za reševanje robnega problema oblike 
\(\frac{dy}{dx}=fun(x,y)\), kjer je \(x\in [a,b]\), podan imamo še \(y(a)\) in korak \(h\).

V datoteki \texttt{EulerImplicitna.m} implementiramo implicitno Eulerjevo metodo.
Interval \([a,b]\) razdelimo na \(n=\frac{b-a}{h}\) delov z delilnimi
točkami \(\{x_i\}_{i=0}^{n}\), \(x_i=a+i\cdot h\). Približke \(y_i\approx y(x_i)\)
izračunamo na naslednji način: 
\[y_i=y_{i-1}+h\cdot fun(x_i,y_i).\]
Formula je implicitna, zato moramo $y_i$ računati s pomočjo navadne
iteracije. Začetni približek za $y_i$, ki ga označimo z $y_i^{(0)}$,
izračunamo s pomočjo eksplicitne Eulerjeve metode:
$$y_i=y_{i-1}+h\cdot fun(x_{i-1},y_{i-1}).$$
Nadaljujemo z računanjem približkov $y_i^{(k)}$, $k=1,2,\ldots$
na naslednji način:
$$y_i^{(k)}=y_{i-1}+h\cdot fun(x_i,y_i^{(k-1)})$$
in za končni približek vzamemo tisti $y_i^{(k)}$, za katerega je
razlika $y_i^{(k)}-y_i^{(k-1)}$ manjša od neke tolerance, v našem 
primeru za toleranco vzamemo kar $10^{-6}$.

V datoteki \texttt{EulerIzboljsana.m} implementiramo izboljšano
Eulerjevo metodo. Interval $[a,b]$ razdelimo na $n=\frac{b-a}{2h}$
delov z delilnimi točkami $\{x_{\frac{i}{2}}\}_{i=0}^n$,
$x_{\frac{i}{2}}=a+\frac{i}{2}h$.  Približke \(y_i\approx y(x_i)\)
izračunamo na naslednji način:
$$y_{i-\frac{1}{2}}=y_{i-1}+\frac{h}{2}fun(x_{i-1},y_{i-1})\textrm{, kjer }
i=1,2,\ldots \textrm{ in }$$
$$y_i = y_{i-1} + h \cdot fun(x_{i-\frac{1}{2}},y_{i-\frac{1}{2}})\textrm{, kjer }
i=1,2,\ldots$$

V datoteki \texttt{Heunova.m} implementiramo Heunovo metodo.
Interval \([a,b]\) razdelimo na \(n=\frac{b-a}{h}\) delov z delilnimi
točkami \(\{x_i\}_{i=0}^{n}\), \(x_i=a+i\cdot h\).
Butcherjeva shema za Heunovo metodo je oblike
\begin{center}
\begin{tabular}{c|cc}
0 & 0 &  \\
$\frac{2}{3}$ & $\frac{2}{3}$ & 0 \\ \hline
& $\frac{1}{4}$ & $\frac{3}{4}$ 
\end{tabular}
\end{center} 
Približke \(y_i\approx y(x_i)\) izračunamo na naslednji način:
$$k_1=fun(x_{i-1},y{i-1})$$
$$k_2=fun(x_{i-1}+\frac{2}{3}h,y_{i-1}+\frac{2}{3}h\cdot k_1)$$
$$y_i=y_{i-1}+h(\frac{1}{4}k_1+\frac{3}{4}k_2).$$

\todo{rezultati}

\section*{2.\ naloga}

Implementacija 2.\ naloge se nahaja v datotekah 
\texttt{BDF.m} in \texttt{RungeKutta4.m}, rezultati pa se preverijo 
v datoteki \texttt{preverjanje2.m}.

V datoteki \texttt{BDF.m} implementiramo implicitno 4-člensko BDF
metodo za robnega problema oblike 
\(\frac{dy}{dx}=fun(x,y)\), kjer je \(x\in [a,b]\), podan imamo še \(y(a)\) in korak \(h\).

Približke $y_n$ pri tej metodi računamo po formuli 
$$\sum_{i=1}^4 \frac{1}{i} \triangledown^i y_n = fun(x_n,y_n).$$
Tu se $\triangledown^i y_n$ izraža kot
$$\triangledown^i y_n = \sum_{j=0}^i (-1)^j {i\choose j} y_{n-j},$$
torej lahko izrazimo 
$$y_n = \frac{1}{\sum_{i=0}^4 \frac{1}{i}}(h\cdot fun(x_n,y_n)-
\sum_{i=1}^4 \frac{1}{i}\sum_{j=1}^i (-1)^j{i \choose j}y_{n-j}).$$
Metoda je implicitna, zato $y_n$ računamo z iteracijo, kot pri 1. nalogi
pri implicitni Eulerjevi metodi. Začetni približek $y_n^(0)$ izračunamo
z Runge-Kutta metodo reda 4, nato pa računamo
$$y_n^(k) = \frac{1}{\sum_{i=0}^4 \frac{1}{i}}(h\cdot fun(x_n,y_n^(k-1))-
\sum_{i=1}^4 \frac{1}{i}\sum_{j=1}^i (-1)^j{i \choose j}y_{n-j}),$$
dokler ni razlika $y_i^{(k)}-y_i^{(k-1)}$ manjša od $10^{-6}$.
Pri implicitni 4-stopenjski BDF metodi moramo najprej določiti 
$y_i$ za $i=0,1,2,3$. Poznamo že $y_0$, $y_1, y_2$ in $y_3$ pa določimo
z Runge-Kutta metodo reda 4, ki ima naslednjo Butcherjevo shemo:
\begin{center}
    \begin{tabular}{c|cccc}
     0&  &  &  &  \\
     $\frac{1}{2}$ & $\frac{1}{2}$ &  &  &  \\
     $\frac{1}{2}$ & 0 & $\frac{1}{2}$ &  &  \\
     1 & 0 & 0 & 1 &  \\ \hline
     $\frac{1}{6}$ & $\frac{2}{6}$ & $\frac{2}{6}$ & $\frac{1}{6}$ & 
    \end{tabular}
\end{center}
To metodo implementiramo v datoteki \texttt{RungeKutta4.m}.
\todo{rezultati}

\section*{3.\ naloga}

Implementacija 3.\ naloge se nahaja v datotekah \texttt{MilneSistem.m}
in \texttt{RungeKutta4.m}. 

\end{document}